%
% Halaman Abstrak
%
% @author  Andreas Febrian
% @version 1.00
%
%\chapter*{Abstrak}
\setstretch{1}
\vspace*{0.2cm}
%    \begingroup
    \singlespacing
	\setlength{\parindent}{0pt}
	
	\begin{tabular}{@{}l l p{10cm}}
		Nama&: & \penulis \\
		Program Studi&: & \program \\
		Judul&: & \judul \\
		Pembimbing&: & \pembimbing \\
	\end{tabular}

	\bigskip
	\bigskip

Adopsi \textit{Virtual Private Network} (VPN) yang masif telah meningkatkan privasi pengguna, namun di sisi lain menimbulkan tantangan signifikan bagi manajemen dan keamanan jaringan dengan menyamarkan konten lalu lintas. Metode analisis tradisional seperti \textit{Deep Packet Inspection} (DPI) menjadi tidak efektif akibat enkripsi, sehingga menciptakan kebutuhan krusial akan teknik yang mampu mengklasifikasikan lalu lintas tanpa mendekripsi \textit{payload}-nya. Meskipun banyak metode dapat mendeteksi keberadaan VPN, klasifikasi rinci (\textit{fine-grained}) terhadap aplikasi atau kategori spesifik di dalam terowongan terenkripsi masih menjadi masalah yang kompleks. Penelitian ini mengusulkan dan memvalidasi sebuah model baru berbasis \textit{flow}, \textit{Hybrid Flow Vector} (HFV), yang dirancang untuk mengatasi tantangan ini pada dataset ISCX 2016. HFV adalah vektor multi-modal yang terdiri dari tiga komponen: ($\alpha$) satu set fitur 128-dimensi yang diekstraksi dari 1D-Convolutional Neural Network (1D-CNN) pada \textit{payload} paket mentah; ($\beta$) satu set fitur 39-dimensi berisi statistik level \textit{flow}; dan ($\gamma$) satu set fitur 37-dimensi yang merinci statistik level \textit{burst}. Studi ablasi pada dataset gabungan (VPN/Non-VPN) memberikan justifikasi untuk utilisasi model penuh ($\alpha$ + $\beta$ + $\gamma$). Model hibrida penuh ini secara konsisten mencapai akurasi tinggi di berbagai tugas klasifikasi, mencatatkan akurasi puncak 97.28\% untuk klasifikasi biner dan 81.88\% untuk klasifikasi kategori, membuktikan bahwa fitur \textit{deep learning} dan statistik bersifat komplementer. Selanjutnya, dengan berfokus pada masalah inti klasifikasi \textit{di dalam} lalu lintas VPN, analisis juga mengungkap bahwa integrasi ketiga komponen fitur menghasilkan kinerja yang paling robust. Untuk klasifikasi kategori VPN, model Hibrida Penuh mencapai performa optimal 94.48\%. Serupa dengan itu, untuk tugas klasifikasi aplikasi spesifik di dalam VPN, model Hibrida Penuh kembali terbukti paling akurat dengan capaian 91.24\%, mengungguli kombinasi subset fitur lainnya. Penelitian ini membuktikan bahwa kerangka kerja HFV menyediakan solusi yang tangguh dan adaptif, di mana integrasi holistik antara analisis \textit{payload} terenkripsi dan statistik perilaku mampu menghasilkan akurasi yang sebanding dengan, atau bahkan melampaui, model-model \textit{state-of-the-art}.


%[\catatan{Sesuaikan ia abstrak berdasarkan ketentuan berikut}: Abstrak merupakan ikhtisar padat dan informatif dari keseluruhan tugas akhir yang esensinya \catatan{memuat permasalahan}, \catatan{tujuan}, \catatan{metode penelitian}, \catatan{hasil}, dan \catatan{kesimpulan}, dirancang untuk memudahkan pembaca memahami secara cepat isi tugas akhir dan memutuskan relevansinya. Abstrak ini harus \catatan{ditulis dalam satu paragraf tunggal} dengan \catatan{panjang maksimum 500 kata}, menggunakan \catatan{font Times New Roman} \catatan{ukuran 12 poin} dengan \catatan{spasi tunggal}. Penting untuk dicatat bahwa abstrak harus disusun dalam dua bahasa, yaitu \catatan{Bahasa Indonesia} dan \catatan{Bahasa Inggris}, di mana setiap versi bahasa mengikuti ketentuan format yang sama dan sedapat mungkin diletakkan dalam satu halaman. Di bagian atas abstrak, harus dicantumkan \catatan{Nama Mahasiswa (tanpa NPM)}, \catatan{Program Studi}, dan \catatan{Judul Tugas Akhir}. Bagian bawah setiap abstrak (baik Bahasa Indonesia maupun Bahasa Inggris) harus diikuti oleh \catatan{Kata Kunci} yang relevan, disajikan dalam bahasa yang sesuai (Bahasa Indonesia untuk abstrak Bahasa Indonesia, dan padanan Bahasa Inggris untuk abstrak Bahasa Inggris). Semua istilah asing, kecuali nama, \catatan{wajib dicetak miring (italic)}. Isi spesifik abstrak dapat disesuaikan dengan keilmuan masing-masing bidang studi.]
	\bigskip

	Kata kunci:	klasifikasi lalu lintas VPN, analisis lalu lintas terenkripsi, klasifikasi berbasis \textit{flow}, \textit{machine learning}, \textit{deep learning}, vektor signature, ISCX 2016.
%    \endgroup

\newpage